\section{Introduction}
As population grows as a result of rapid development of cities, so does the volume of traffic and congestion.
Managing traffic congestion is a serious issue, and  with the ever-increasing volume of traffic, this problem is expected to get worse.\\
A simple description of traffic congestion is given as a situation where the volume of traffic surpasses the capacity of the road. However, the Department for Transport provide a more quantifiable description of congestion as "the average delay experienced for each kilometre travelled compared to driving at speeds typical when traffic is light" \cite{2}.
A lot of attention has been given to modelling and simulation of traffic flows to determine the causes of traffic congestion, to determine the effectiveness of traffic policies  and how such road policies can be improved and also aid in the development and design or road infrastructures.

\section*{Problem Definition}
Many strategies have been proposed and implemented with the primary objective of relieving congestion. However, these strategies have varying effects and it is not always self-evident which strategy works best for a given scenario.\\
A straightforward strategy is to construct new roads to increase the road capacity or improve already existing roads, but  this rarely serves as a long time solution.
Congestion is described as self managing, that is; as the capacity of the road increases, so does the traffic demand to fill the new capacity. This notion is described as Pigou-Knight analysis, which suggests improving road network capacity does not guarantee reduced congestion, on the contrary it can be counter-productive by making congestion worse\cite{1,2}. \\ This strategy regarded as one of the most expensive ways of dealing with congestion \cite{2}.
\par{}
Traffic simulation is important because it allows very complex traffic models to be studied and analysed. In situations where evaluating analytical and numerical data is not sufficient enough it provides visual description, of both existing and future scenarios.
Another huge importance of traffic simulation is, it enables us see the outcome changes to road infrastructure will have.

\section*{Project Summary}
In this report, we give a detailed documentation of the process involved in developing a traffic simulation software. \\
Our traffic simulation software consist of two parts: the simulation environment and the map builder. \\
The first part of the system is the simulation environment where actual simulation runs. This part allows a user to parse a map, which is then loaded and executed.\\
The second part of the system is the map maker (builder). This part of our system allows the user to create a map. That is, it serves as an interface which provides all the necessary functionalities for designing road infrastructure (i.e lanes, intersections), which can then be parsed, loaded and executed in the simulation environment.
\par{}
The rest of this report is structured as follows. In section 2, we review the various literature related to traffic simulation. Furthermore, we also review existing works related in this domain. We proceed to section 3 where we provide the description of our initial project requirements  and the design approach we have chosen.